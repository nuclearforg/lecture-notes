%!TEX root = ../mst.tex

\chapter{Introduction}
A \textbf{homogeneous system} is a system in which chemical and physical
properties are independent of the position (uniform).

a \textbf{heterogeneous system} is a system in which chemical and physical
properties depend on the position. A particular set of heterogeneous systems are
multiphase systems.

A \textbf{multiphase system} can be thought as a heterogeneous system coming
from the union of homogeneous systems. Each homogeneous portion is called
\textbf{phase}.

% TODO figure: phases

Phases are separated by boundaries, called \textbf{interfaces}. Interface are
usually:
\begin{itemize}
    \item deformable (moving)
    \item diathermal (diabatic)
    \item permeable to chemical species
\end{itemize}

From the point of view of energy, an interface usually allows the exchange of
work, heat and chemical energy.

\section{Thermodynamic equilibrium}
\subsection{How many phases can coexist at equilibrium?}
Let's consider a system with $M$ phases and $r$ chemical components (chemical
species). Then let's write equilibrium equations:
\begin{itemize}
    \item \textbf{Mechanical equilibrium}: for planar interfaces\footnote{The
    number of equations is independent of the shape of the interface, but for
    the sake of simplicity we will consider flat boundaries. We will extend
    these considerations when dealing with bubbles or drops.} pressure $p$ is
    uniform
    \begin{equation*}
        p^{(1)} = p^{(2)} = \dots = p^{(M)}
    \end{equation*}
    \item \textbf{Thermal equilibrium}: temperature $T$ is uniform
    \begin{equation*}
        T^{(1)} = T^{(2)} = \dots = T^{(M)}
    \end{equation*}
    \item \textbf{Chemical equilibrium}: we have $r\times M$ equations, chemical
    potential is the same throughout the heterogeneous system
    \begin{align*}
        \mu_1^{(1)} &= \mu_1^{(2)} = \dots = \mu_1^{(M)} \\
        \mu_2^{(1)} &= \mu_2^{(2)} = \dots = \mu_2^{(M)} \\
        &\vdots \\
        \mu_r^{(1)} &= \mu_r^{(2)} = \dots = \mu_r^{(M)}
    \end{align*}
\end{itemize}

This system of equations states the thermodynamic equilibrium and contains
$M\times(r+2)$ variables and $(M-1)\times(r+2)$ equations.

Each one of the $M$ phases is an homogeneous system by itself, hence each phase
has to satisfy the Gibbs-Duhem Equation to fullfill the first and second
principles of thermodynamics:
\begin{equation*}
    SdT-Vdp+\sum_{i=1}^r N_id\mu_i = 0
\end{equation*}
The real number of equations now becomes $(M-1)\times(r+2) + M$.

Given that the well-posedness of a system arises when $\#\text{ Variables} \ge
\#\text{ Equations}$,
\begin{align*}
    \#\text{ Equations} &\le \#\text{ Variables} \\
    (M-1)(r+2)+M &\le M(r+2) \\
    M(r+2)-r-2+M &\le M(r+2) \\
    M &\le r+2
\end{align*}

\subsubsection{Examples}
In the case $r=1$ (single component system),
\begin{equation*}
    r=1 \implies M \le 3 \quad
    \begin{cases}
        M = 3 \to \text{ three phase system (triple state)} \\
        M = 2 \to \text{ two phase system} \\
        M = 1 \to \text{ single phase system}
    \end{cases}
\end{equation*}

In the case $r=2$ (two component system), $r=2 \implies M \le 4$. An example
could be moist air, given that we approximate dry air as if it was made by a
single component. In a lab environment could be possible to observe a mixture of
condensed air, condensed water and moist air.

\subsection{How many quantities are required to describe equilibrium?}
Let’s now derive the number of indipendent variables (also known as
\emph{degrees of freedom}):
\begin{align*}
    f &= \#\text{ Variables} - \#\text{ Equations} \\
    &= M(r+2) - (M-1)(r+2) - M \\
    &= M(r+2) - M(r+2) + r+2 - M \\
    &= r - M + 2
\end{align*}

\subsubsection{Examples}
For a single component system ($r=1$) the degrees of freedom are $f=3-M$. Three
cases:
\begin{itemize}
    \item $M=3 \implies f=0$ meaning that we cannot set any quantity among
    temperature, pressure and chemical potential;
    \item $M=2 \implies f=1$ meaning we are only allowed to set one variable,
    typically the temperature or the pressure (\emph{saturation condition});
    \item $M=1 \implies f=2$ meaning we are free to set temperature and pressure
    independently.
\end{itemize}

\section{Basic quantities}